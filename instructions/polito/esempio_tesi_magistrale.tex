% per commentare una riga mettere % al suo inizio
% per s-commentare una riga (ossia attivarla) togliere il % al suo inizio
%
\documentclass[pdfa% formato PDF/A, obbligatorio per l'archiviazione delle tesi di Polito
,cucitura%lascia margine per la rilegatura
%,twoside% per stampa fronte-retro (fortemente consigliato per tesi voluminose, opzionale per le altre)
%,12pt% font pi� grande (12pt) rispetto a quello normalmente usato (11pt)
]{toptesi}
%
% Commentare le righe seguenti se NON si � specificata l'opzione "pdfa"
\hypersetup{%
    pdfpagemode={UseOutlines},
    bookmarksopen,
    pdfstartview={FitH},
    colorlinks,
    linkcolor={blue},
    citecolor={red},
    urlcolor={blue}
  }
% \documentclass[11pt,twoside,oldstyle,autoretitolo,classica,greek]{toptesi}
% \usepackage[or]{teubner}
%%%%%%%%%%%%%%%%%%%%%%%%%%%%%%%%%%%%%%%%%%%%%%%%%%%%
%
% Esempio di composizione di tesi di laurea.
%
% Questo esempio e' stato preparato inizialmente 13-marzo-1989
% e poi e' stato modificato via via che TOPtesi andava
% arricchendosi di altre possibilita'.
%
% Nel seguito laurea "quinquennale" sta anche per "specialistica" o "magistrale"

% Cambiare encoding a piacere; oppure non caricare nessun encoding se si usano
% solo caratteri a 7 bit (ASCII) nei file d'entrata.
%
\usepackage[latin1]{inputenc}% IMPORTANTE! usare codifica ISO-8859-1 per le lettere accentate

\ateneo{Politecnico di Torino}

%%% scegliere la propria facolt� (solo PRIMA dell'AA 2012-2013)
%
%\facolta[III]{Ingegneria dell'Informazione}
%\facolta[IV]{Organizzazione d'Impresa\\e Ingegneria Gestionale}
%\Materia{Remote sensing}% uso sconsigliato

%\monografia{Gestione informatizzata di un magazzino ricambi}% per la laurea triennale
\titolo{Esempio di tesi (v. 1.01)}% per la laurea quinquennale e il dottorato
%\sottotitolo{Metodo dei satelliti medicei}% NON obbligatorio, per la laurea quinquennale e il dottorato

%%% scegliere il proprio corso
%
%\corsodilaurea{Ingegneria dell'Organizzazione d'Impresa}% per la laurea di primo e secondo livello
%\corsodilaurea{Ingegneria Logistica e della Produzione}% per la laurea di primo e secondo livello
%\corsodilaurea{Ingegneria Gestionale}% per la laurea di primo e secondo livello
\corsodilaurea{Ingegneria Informatica}% per la laurea di primo e secondo livello
%\corsodidottorato{Meccanica}% per il dottorato

\candidato{Giovanni \textsc{Pautasso}}% per tutti i percorsi
%\secondocandidato{Evangelista \textsc{Torricelli}}% per la laurea magistrale solamente
%\direttore{prof. Albert Einstein}% per il dottorato
%\coordinatore{prof. Albert Einstein}% per il dottorato
\relatore{prof.\ Antonio Lioy}% per la laurea e il dottorato
%\secondorelatore{dipl.~ing.~Werner von Braun}% per la laurea magistrale
%\terzorelatore{{\tabular{@{}l}dott.\ Neil Armstrong\\prof. Maria Rossi\endtabular}}% per la laurea magistrale
%\tutore{ing.~Karl Von Braun}% per il dottorato
%\tutoreaziendale{dott.\ ing.\ Giovanni Giacosa} % solo per la laurea di secondo livello con tesi svolta in azienda
%\NomeTutoreAziendale{Supervisore aziendale\\Centro Ricerche FIAT}
%\sedutadilaurea{Agosto 1615}% per la laurea quinquennale
%\esamedidottorato{Novembre 1610}% per il dottorato
\sedutadilaurea{\textsc{Marzo} 2013}% per la laurea triennale
%\sedutadilaurea{\textsc{Anno~accademico} 1615-1616}% per la laurea magistrale
%\annoaccademico{1615-1616}% solo con l'opzione classica
%\annoaccademico{2006-2007}% idem
%\ciclodidottorato{XV}% per il dottorato
\logosede{../../img/logopolito}
%
%\chapterbib %solo per vedere che cosa succede; e' preferibile comporre una sola bibliografia
%\AdvisorName{Supervisors}
%\newtheorem{osservazione}{Osservazione}% Standard LaTeX

%\usepackage[a-1b]{pdfx}
%\hypersetup{%
%    pdfpagemode={UseOutlines},
%    bookmarksopen,
%    pdfstartview={FitH},
%    colorlinks,
%    linkcolor={blue},
%    citecolor={green},
%    urlcolor={blue}
%  }

%
% per numerare e far comparire nell'indice anche le sezioni di quarto livello
% SCONSIGLIATO! da usarsi solo in caso di estrema necessit�
%\setcounter{secnumdepth}{4}% section-numbering-depth
%\setcounter{tocdepth}{4}% TOC-numbering-depth (TOC=Table-Of-Content)

%\setbindingcorrection{3mm}

\input{commands.tex}

\begin{document}

\errorcontextlines=9

\expandafter\ifx\csname StileTrieste\endcsname\relax
    \frontespizio
\else
    \paginavuota
    \begin{dedica}
        A mio padre

        \textdagger\ A mio nonno Pino
    \end{dedica}
    \tomo
\fi


\sommario

Inserire qui un breve sommario della tesi.

\ringraziamenti

Opzionali, solo nel caso si sia ricevuto un aiuto speciale e particolarmente rilevante.

%% inserire sempre nella tesi per la laurea di I livello, perch� il nome dei tutori non � indicato sul frontespizio.
%Il lavoro descritto in questa monografia � stato svolto sotto la supervisione
%del Prof. Antonio Lioy (tutore accademico)% inserire sempre il nome del tutore accademico
% e dell'Ing. Mario Rossi (tutore aziendale)% inserire solo se la monografia � relativa ad un tirocinio.
%.

%\tablespagetrue % normalmente questa riga non serve ed e' commentata
%\figurespagetrue % normalmente questa riga non serve ed e' commentata

\indici

\mainmatter

\chapter{Scrivere testi complessi}

\input{latex.tex}

\chapter{La tesi di laurea magistrale}

\input{magistrale.tex}

\chapter{Progettazione}

Discutere in questo capitolo come � stata progettata la soluzione al problema trattato nella tesi, indicando anche se sono stati valutati vari possibili approcci o soluzioni pre-esistenti e giustificando le proprie scelte. Descrivere quindi la soluzione vera e propria.

Nel caso sia stato sviluppato del software non triviale, � buona norma dedicargli tre sezioni:
\begin{itemize}
\item architettura dell'applicazione (interazioni con gli utenti e con altri sistemi, moduli logici, flussi dati interni ed esterni);
\item manuale dello sviluppatore (descrizione dei moduli, degli algoritmi, delle interfacce e delle strutture dati);
\item manuale utente (come installare ed usare il programma, interfacce, comandi, dati in input ed in output).
\end{itemize}
Nel caso di software molto voluminoso, queste tre sezioni possono diventare tre capitoli separati.

\chapter{Risultati}

Inserire in questo capitolo i risultati conseguiti, cercando di analizzarli -- se possibile -- in modo quantitativo.


\chapter{Conclusioni}

Qui si inseriscono brevi conclusioni sul lavoro svolto, senza ripetere inutilmente il sommario.

Si possono evidenziare i punti di forza e quelli di debolezza, nonch� i possibili sviluppi futuri o attivit� da svolgere per migliorare i risultati.


% La bibliografia, da inserirsi solo se ci sono state citazioni.
% In questo caso ricordarsi che bisogna sempre elaborare due volte il file .TEX
% perch� la prima volta viene generata la bibliografia mentre la seconda volta viene inclusa

% NOTA: citare il DOI non � obbligatorio ma MOLTO desiderabile

\begin{thebibliography}{9} % se ci sono meno di 10 citazioni
%\begin{thebibliography}{99} % se ci sono da 10 a 99 citazioni

\bibitem{psisec} % esempio citazione articolo a congresso
I.Enrici, M.Ancilli, A.Lioy, % nomi autori
``A psychological approach to information technology security'', % titolo articolo
HSI-2010: 3rd Int. Conf. on Human System Interactions, % nome del congresso
Rzesz�w (Poland), May 13-15, 2010, % luogo (stato) e data del congresso
pp.\ 459-466, % pagine dell'articolo
\doi{10.1109/HSI.2010.5514528}

\bibitem{tpa} % esempio citazione articolo su rivista
G.Cabiddu, E.Cesena, R.Sassu, D.Vernizzi, G.Ramunno, A.Lioy,  % autori dell'articolo
``Trusted Platform Agent'', % titolo dell'articolo
IEEE Software, % nome della rivista
Vol.\ 28, No.\ 2, % volume e numero della rivista (alcune riviste non ce l'hanno)
March-April 2011, % mese e anno di pubblicazione della rivista
pp.\ 35-41, % pagine dell'articolo
\doi{10.1109/MS.2010.160}


\bibitem{tc} % esempio citazione capitolo di un libro fatto come collezione di contributi da autori diversi
A.Lioy, G.Ramunno, % autori del capitolo
``Trusted Computing'' % titolo del capitolo
nel libro % in the book
``Handbook of Information and Communication Security'' % titolo del libro
a cura di % edited by
P.Stavroulakis, M.Stamp, % nomi dei curatori
Springer, % nome editore
2010, % anno di pubblicazione
pp.\ 697-717, % pagine del capitolo
\doi{10.1007/978-3-642-04117-4_32}

\bibitem{openssl} % esempio citazione pagina web di un progetto
The OpenSSL project, % nome del progetto
\url{http://www.openssl.org/} % URI della pagina web

\bibitem{tls12} % esempio citazione RFC
T.Dierks, E.Rescorla,
``The Transport Layer Security (TLS) Protocol Version 1.2'',
\rfc{5246}, August 2008

\bibitem{seceng} % esempio citazione libro
Ross J. Anderson,
``Security engineering'',
Wiley, 2008,
ISBN: 978-0-470-06852-6


\end{thebibliography}



\end{document}
