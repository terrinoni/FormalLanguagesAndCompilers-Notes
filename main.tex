%%%%%%%%%%%%%%
%% Run LaTeX on this file several times to get Table of Contents,
%% cross-references, and citations.

%% If you have font problems, you may edit the w-bookps.sty file
%% to customize the font names to match those on your system.

%% w-bksamp.tex. Current Version: Feb 16, 2012
%%%%%%%%%%%%%%%%%%%%%%%%%%%%%%%%%%%%%%%%%%%%%%%%%%%%%%%%%%%%%%%%
%
%  Sample file for
%  Wiley Book Style, Design No.: SD 001B, 7x10
%  Wiley Book Style, Design No.: SD 004B, 6x9
%
%
%  Prepared by Amy Hendrickson, TeXnology Inc.
%  http://www.texnology.com
%%%%%%%%%%%%%%%%%%%%%%%%%%%%%%%%%%%%%%%%%%%%%%%%%%%%%%%%%%%%%%%%

%%%%%%%%%%%%%
% 7x10
\documentclass{wileySev}

% 6x9
%\documentclass{wileySix}

\usepackage{graphicx}

%%%%%%%
%% for times math: However, this package disables bold math (!)
%% \mathbf{x} will still work, but you will not have bold math
%% in section heads or chapter titles. If you don't use math
%% in those environments, mathptmx might be a good choice.

% \usepackage{mathptmx}

% For PostScript text
\usepackage{w-bookps}

\usepackage{listings}
\lstset{
    basicstyle=\ttfamily,
    columns=fullflexible,
    xleftmargin=3ex,
    breaklines,
    breakatwhitespace,
    escapechar=`
}

\newcommand{\code}[1]{\lstinline|#1|}

%%%%%%%%%%%%%%%%%%%%%%%%%%%%%%%%%%%%%%%%%%%%%%%%%%%%%%%%%%%%%%%%
%% Other packages you might want to use:

% for chapter bibliography made with BibTeX
% \usepackage{chapterbib}

% for multiple indices
% \usepackage{multind}

% for answers to problems
% \usepackage{answers}

%%%%%%%%%%%%%%%%%%%%%%%%%%%%%%
%% Change options here if you want:
%%
%% How many levels of section head would you like numbered?
%% 0= no section numbers, 1= section, 2= subsection, 3= subsubsection
%%==>>
\setcounter{secnumdepth}{3}

%% How many levels of section head would you like to appear in the
%% Table of Contents?
%% 0= chapter titles, 1= section titles, 2= subsection titles, 
%% 3= subsubsection titles.
%%==>>
\setcounter{tocdepth}{2}

%% Cropmarks? good for final page makeup
%% \docropmarks

%%%%%%%%%%%%%%%%%%%%%%%%%%%%%%
%
% DRAFT
%
% Uncomment to get double spacing between lines, current date and time
% printed at bottom of page.
% \draft
% (If you want to keep tables from becoming double spaced also uncomment
% this):
% \renewcommand{\arraystretch}{0.6}
%%%%%%%%%%%%%%%%%%%%%%%%%%%%%%

%%%%%%% Demo of section head containing sample macro:
%% To get a macro to expand correctly in a section head, with upper and
%% lower case math, put the definition and set the box 
%% before \begin{document}, so that when it appears in the 
%% table of contents it will also work:

\newcommand{\VT}[1]{\ensuremath{{V_{T#1}}}}

%% use a box to expand the macro before we put it into the section head:

\newbox\sectsavebox
\setbox\sectsavebox=\hbox{\boldmath\VT{xyz}}

%%%%%%%%%%%%%%%%% End Demo


\begin{document}

    \booktitle{Formal Languages and\\Compilers}
    \subtitle{Course notes}

    \authors{Marco Terrinoni\\\affil{Polytechnic University of Turin}}

    %\offprintinfo{Survey Methodology, Second Edition}{Robert M. Groves}

    %% Can use \\ if title, and edition are too wide, ie,
    %% \offprintinfo{Survey Methodology,\\ Second Edition}{Robert M. Groves}

    %%%%%%%%%%%%%%%%%%%%%%%%%%%%%%
    %% 
    %\halftitlepage

    \titlepage

    %\dedication{To my parents}

    \begin{contributors}
    %   \name{Masayki Abe,} Fujitsu Laboratories Ltd., Fujitsu Limited, Atsugi, Japan

    %   \name{L. A. Akers,} Center for Solid State Electronics Research, Arizona State University, Tempe, Arizona

    %   \name{G. H. Bernstein,} Department of Electrical and Computer Engineering, University of Notre Dame, Notre Dame, South Bend, Indiana; formerly of Center for Solid State Electronics Research, Arizona State University, Tempe, Arizona 
    \end{contributors}

    %\contentsinbrief
    \tableofcontents
    %\listoffigures
    %\listoftables

    \begin{preface}
        This book collects the various personal notes from the course ``Formal Languages and Compilers''.\\
        In case of errors or additional material, please contact me at my private email address\\
        \code{marco.terrinoni90@gmail.com}

        \where{Turin, Italy\\
        September, 2014}
    \end{preface}

    \begin{acronyms}
        \acro{FLC}{Formal Languages Classification}
        \acro{RL}{Regular Languages}
        \acro{CFL}{Context-Free Language}
        \acro{TM}{Turing Machines}
    \end{acronyms}

    %\begin{glossary}
    %   \term{NormGibbs}Draw a sample from a posterior distribution of data with an unknown mean and variance using Gibbs sampling.
    %   \term{pNull}Test a one sided hypothesis from a numerically specified posterior CDF or from a sample from the posterior
    %   \term{sintegral}A numerical integration using Simpson's rule
    %\end{glossary}

    \part[Languages]
    {Languages}

        \chapter[Classification (FLC)]
        {Classification (FLC)}
        \section{Grammars}
A grammar is a 4-tuple $G = (N, T, P, S)$ where:
\begin{description}
	\item[N] alphabet of non-terminal symbols;
	\item[T] alphabet of terminal symbols;
	\item[P] finite set of rules (productions);
	\item[S] start (non-terminal) symbol.
\end{description}

A language produced by $G = (N, T, P, S)$ is:
$$
	L(G) = \left\{w \middle| w \in T\ast ; S \Rightarrow\ast w\right\}
$$

        \chapter[Regular Languages (RL)]
        {Regular Languages (RL)}

        \chapter[Context-Free Languages (CFL)]
        {Context-Free Languages (CFL)}

        \chapter[Turin Machines (TM)]
        {Turin Machines (TM)}

    \part[Compilers]
    {Compilers}

        \chapter[Compiler Structure (CS)]
        {Compiler Structure (CS)}

        \chapter[Lexical Analysis (LA)]
        {Lexical Analysis (LA)}

        \chapter[Syntax Analysis (SA)]
        {Syntax Analysis (SA)}

        \chapter[Syntax-Directed Translation (SDT)]
        {Syntax-Directed Translation (SDT)}

        \chapter[Semantic Analysis and Intermediate-Code Generation (SA/ICG)]
        {Semantic Analysis and Intermediate-Code Generation (SA/ICG)}

\end{document}
