\section{Deterministic Finite Automata (DFA)}
A DFA is a 5-tuple $A = (Q, \Sigma, \delta, q_0, F)$ where:
\begin{description}
	\item[$Q$] finite (non-empty) set of states;
	\item[$\Sigma$] alphabet of input symbols;
	\item[$\delta$] transition function:
		$$
			\delta: Q \times \Sigma \to Q
		$$
	\item[$q_0$] start state:
		$$
			q_0 \in Q
		$$
	\item[$F$] set of final states:
		$$
			F \subseteq Q
		$$
\end{description}

\subsection{Transition Table}
Transitional Table is a tabular representation of this transition function.

\subsection{Transition Diagram}
Transitional Diagram is a graph where:
\begin{itemize}
	\item for each state in the automaton there a node;
	\item for each transition $\delta(p, a) = q$ there is an arc from $p$ to $q$ labelled $a$.
\end{itemize}
The start state has an entering non-labelled arc and the final states are marked by a double circle.

\section{Non-Deterministic Finite Automata (NFA)}
An NFA is a 5-tuple 

\section{Equivalence of NFA and DFA}

\section{From Finite Automata to Regular Expression}