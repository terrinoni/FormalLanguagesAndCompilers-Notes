The advanced usage of CUP in this chapter includes:
\begin{itemize}
    \item
    grammars with ambiguities;
    \item
    lists;
    \item
    operator precedence;
    \item
    handling syntax errors.
\end{itemize}

\section{Ambiguous Grammars in CUP}
Conflicts can arise when the grammar is ambiguous; this implies that the parser must choose between two or more alternative actions.
The problem can be resolved by modifying the grammar (in order to make it non-ambiguous) or by instructing the parser on how to handle ambiguity.
The latter option requires that the parsing algorithm is fully understood, in order to avoid unwanted/wrong behaviors.

A grammar is ambiguous if there is at least one sequence of symbols for which two or more distinct parse trees exist.
Example:
\begin{lstlisting}
import java.io.*;

public class Main {
    static public void main(String argv[]) {
        try {
            // instantiate the scanner and open input file argv[0]
            Yylex L = new Yylex(new FileReader(argv[0]));
            // instantiate the parser
            Parser p = new Parser(L);
            // start the parser
            Object result = p.parse();
        } catch(Exception e) {
            e.printStackTrace();
        }
    }
}
\end{lstlisting}