\section{Role of a Parser}
The parser obtains a string of tokens from the scanner and:
\begin{itemize}
	\item verifies that the string can be generated by the grammar for the source language, trying to build a parse tree;
	\item reports syntax errors and continues processing the input.
\end{itemize}
There are two different types of parser:
\begin{itemize}
	\item \emph{Bottom-Up} parsers build parse trees from the bottom (leaves) to the top (root);
	\item \emph{Top-Down} parsers build parse trees from the top (root) to the bottom (leaves).
\end{itemize}

\section{Bottom-Up Parsing}
Bottom-Up parsing attempts to construct a parse tree for an input string beginning at the leaves (the bottom).
this construction process reduces an input string to the start symbol of a grammar; at each reduction step the right side of a production is replaced by its left side symbol, tracing out a rightmost derivation in reverse.

\section{Handles}
If a string $\alpha \beta w$ can be produced by a rightmost derivation $S \Rightarrow^\ast_{RM} \alpha AW \Rightarrow^\ast_{RM} \alpha \beta w$, then $A \to \beta$ is an handle of $\alpha \beta w$ ($w \in T^\ast$ because $A \to \beta$ is the last applied rule).

\section{Actions of a Shift-Reduce Parser}
\begin{description}
	\item[Shift]\mbox{}\\
	the next input symbol is shifted onto the top of the stack;
	\item[Reduce]\mbox{}\\
	the left end of the handle must be located within the stack; it must be decided with what non-terminal to replace the handle;
	\item[Accept]\mbox{}\\
	parsing is successfully completed;
	\item[Error]\mbox{}\\
	a syntax error has occurred.
\end{description}
A strategy for making parsing decision is needed in order to avoiding conflicts with decision between shift and reduce operations.

\section{LR Parser}
LR parsers are a type of bottom-up parsers that handle deterministic context-free languages in guaranteed linear time.
The \emph{LALR} parsers and the \emph{SLP} parsers are common variants of LR parsers.
LR parsers are of ten generated from a formal grammar for the language by a parser generator tool.

The name LR is an acronym: the \emph{L} means that the parser reads input text in one direction without baking up; that direction is typically ``left-to-right'' within each line, and ``top-to-bottom'' across the lines of the full input file.
The \emph{R} means that the parser produces a reversed rightmost derivation; it does a bottom-up parse.

The name LR is often followed by a numeric qualifier, as in $LR(1)$ or sometimes $LR(k)$.
To avoid backtracking or guessing, the LR parser is allowed to peek ahead a $k$ lookahead input symbols before deciding how to parse earlier symbols.

\subsection{$LR(0)$ Items}
An $LR(0)$ item of a CFG grammar $G$ is a production of $G$ with a dot at some position of the right side; an item indicate how much of a production we have seen at a given point in the parsing process.
The dot indicates the current position of the parser and an item with the dot at the end is called \emph{complete} (that is all the right side of the production has been recognized).

\subsection{Viable Prefixes}
A viable prefix of a string $\gamma$ is a prefix that can appear on the stack of a shift reduce parser.

To recognize viable prefixes:
\begin{itemize}
	\item the sets of viable prefixes are regular languages;
	\item the FA that recognizes them can guide a parser in making decisions;
	\item the valid $LR(0)$ items of a CFG grammar are the states of an NFA recognizing viable prefixes;
	\item a DFA equivalent to such NFA will have states corresponding to sets of $LR(0)$ items and transitions labelled by symbols in viable prefixes.
\end{itemize}

The function $closure(I)$ finds the set of $LR(0)$ items that recognizes the same viable prefix.
The function $goto(I, x)$ finds the set of $LR(0)$ items that is reached from the set $I$ with symbol $x$.

Given a CFG grammar $G = (N, T, P, S)$, the function $items(G)$ constructs the collection $C = \left\{I_0, I_1, \ldots, I_m\right\}$ of DFA states.

The function $lr_0table(G)$ constructs the $LR(0)$ parsing table for the CFG $G$.
The initial state of the parser is the one constructed from the set of items containing $S' \to \bullet S$.

\subsection{Conflicts in Parsing Tables}
Parsing table entries defined in multiple ways determines parsing action conflicts:
\begin{description}
	\item[Shift/Reduce]\mbox{}\\
	some entry in the action table contains both a shift and reduce action;
	\item[Reduce/Reduce]\mbox{}\\
	some entry in the action table contains more reduce actions.
\end{description}

\subsection{$LR(0)$ Grammars}
A grammar $G$ is $LR(0)$ if the action table generated by the function $lr_0table(G)$ does not comprise conflicts.
$LR(0)$ grammars are ``non-ambiguous''.

\subsection{$LR(0)$ Parsers}
An $LR(0)$ parser:
\begin{itemize}
	\item scans the input from left to right;
	\item constructs a rightmost derivation in reverse;
	\item uses zero lookahead input symbols in making parsing decision.
\end{itemize}
The class of languages that can be parsed using $LR(0)$ parsers is a proper subset of the deterministic CFL.

\subsection{$LR(k)$ Parsers}
More powerful parsers can be constructed when more than zero lookahead input symbols are used in making parsing decision.

\subsection{First and Follow Sets}
With respect to a CFG, given a non-terminal symbol $x$ and a string $\gamma$ of terminal and non-terminal symbols:
\begin{description}
	\item[$nullable(x):$] is true if $x$ can derive an empty string;
	\item[$nullable(\gamma):$] is true if each symbol in $\gamma$ is nullable;
	\item[$first(\gamma):$] is the set of terminals that can begin strings derived from $\gamma$;
	\item[$follow(\gamma):$] is the set of terminals that can immediately follow $x$.
\end{description}

\subsection{Simple LR Parsing Table}
The function $slrTable(G)$ constructs the SLR parsing table from the CFG $G$.

\subsection{$LR(k)$ parsers (bis)}
$Follow(A)$ is the set of terminals that can immediately follow A in any string generated by a given grammar $G$.
It takes into account all the contexts where A can appear; by taking into account the specific context of $A$ when the rule $A \to \alpha$ is applied, it could be possible to set $A$ \underline{reduce} $A \to \alpha$ action for a subset of $follow(A)$, thus avoiding further potential conflicts.

\subsection{$LR(1)$ Items}
An $LR(1)$ item of a context free grammar $G$ is a production of $G$ with a dot at some position of the right side and a lookahead (terminal of $\$$) symbol.

An $LR(1)$ item $[A \to \alpha; a]$ calls for a reduction by $A \to \alpha$ only if the next input symbol is $a$ ($follow(A) = a$).

We say item $[A \to \beta_1 \bullet \beta_2, a]$ is valid for a viable prefix $\alpha\beta_1$ if:
\begin{itemize}
	\item there is a derivation $S \Rightarrow^\ast_{RM} aAw \Rightarrow^\ast_{RM} \alpha\beta_1\beta_2w$;
	\item either $a$ is the first symbol of $w$, or $w$ is $\varepsilon$ and $a$ is $\$$.
\end{itemize}
The valid $LR(1)$ items of a CFG are the states on an NFA recognizing viable prefixes.
A DFA equivalent to such NFA will have states corresponding to sets of $LR(1)$ items and transitions labelled by the symbols of the viable prefixes.

The function $closure_1(I)$ finds the set of $LR(1)$ items that recognize the same viable prefix; the function $goto_1(I, x)$ finds the set of $LR(1)$ items that is reached from the set $I$ with symbol $x$.
Given a CFG $G = (N, T, P, S)$, the function $items_1(G)$ constructs the collection $C = \left\{I_0, I_1, \ldots, I_n\right\}$ of DFA states.
The function $lr_1table(G)$ constructs the $LR(1)$ parsing table for the CFG $G$.

\subsection{$LR(1)$ Parsers}
An $LR(1)$ parser:
\begin{itemize}
	\item scans the input from left to right;
	\item constructs a rightmost derivation in reverse;
	\item uses one lookahead input symbol in making parsing.
\end{itemize}
The class of languages that can be parsed using $LR(1)$ parsers is exactly the class of deterministic CFL.

\subsection{Lookahead $LR(1)$ parsers (LALR)}
$LR(1)$ parsing tables can be very large (several thousand states) for grammars generating common programming languages; SLR parsing tables for the same languages are much smaller, but can contain \underline{conflicts}.
$LALR(1)$ parsing tables have the same states of SLR tables and can conveniently express more programming languages.

\subsection{$LALR(1)$ Parsing Tables}
Two sets of $LR(1)$ items have the same core if they are identical except for the lookahead symbols.
A set of $LALR(1)$ items is the union of sets of $LR(1)$ items having the same core.

The merging of states with common cores can never produce a ``shift/reduce'' conflict which was not present in one of the original states (shift actions depend only on the core, not the lookahead); anyway some ``reduce/reduce'' conflicts can occur after merging.

The class of languages that can be parsed using $LALR(1)$ parsers is a proper subset of the deterministic CFL.

\subsection{Using Ambiguous Grammars}
Ambiguous grammars are not $LR(k)$.
Some ambiguous grammars provide shorter and more natural specifications that any equivalent unambiguous grammar.
In some cases disambiguating rules, such as precedence and associativity, can be specified; the resulting parser can be more efficient.
Ambiguous constructs should be used sparingly and in a strictly controlled fashion.

\subsection{Error Recovery in LR Parsing}
Blanks in LR parsing tables mean error actions and cause the parser to stop.
This behaviour would be unkind to the user, who would like to have all the errors reported, not just the first one.
Local error recovery mechanism use a special error symbol to allow parsing to resume; whenever the error symbol appears in a grammar rule, it can match a sequence of erroneous input symbols.

\subsection{Recovery Using the Error Symbol}
Let $A \to error \quad \alpha$ be a grammar production; in the construction of the parsing table:
\begin{itemize}
	\item $error$ is considered a terminal symbol;
	\item error productions are treated as ordinary production.
\end{itemize}
On encountering an error action (a blank in the table), the parser:
\begin{enumerate}
	\item pops the stack until a state is reached where the action for error is shift (a state including an item $A \to \bullet error \quad \alpha$);
	\item shifts a fictitious $error$ token out the stack, as though $error$ was found on input;
	\item skips ahead on the input discarding symbols until a substring is found that can be reduced to $\alpha$;
	\item reduces the handle $error \quad \alpha$ (at this point on top of the stack) to $A$.;
	\item emits a diagnostic message;
	\item resumes normal parsing.
\end{enumerate}
Error rules may introduce both ``shift/reduce'' and ``reduce/reduce'' conflicts; they cannot be inserted anywhere into an LALR grammar,.
This error recovery mechanism is not powerful enough to correctly report all syntactic errors.

\subsection{LR Parser Generators}
An LR parser generator transforms the specification of a parser into a program implementing an LR parser.
CUP produces Java programs implementing $LALR(1)$ parsers.

\section{Top-Down Parsing}
Top-down parsing attempts to construct a parse tree for an input string starting at the root and working down towards the levaes.
This construction process creates the nodes of the tree in pre-order until it obtains the input string.
At each creation step the left side symbol of a production is replaced by its right side, tracing out a leftmost derivation.

\section{Left-Recursive Grammars}
A production like $A \to A\alpha$ is called ``left-recursive production''; a grammar is left-recursive if it can generate a derivation $A \Rightarrow^\ast A\alpha$.
A left-recursive grammar can cause a top-down parser to go into an infinite loop.
Left-recursive productions can be replaced by right-recursive productions:
%% the following math section generates some problems but the output is good...
$$
\text{problem here with math environment}
%	\[
%		A \to A\alpha | \beta \equiv \left\{
%		\begin{array}{c}
%			A \to \beta R \\
%			R \to \alpha R
%		\end{array}
%		\right.
%	\]
$$
($\beta$ doesn't start with $A$)
$$
	A \Rightarrow^\ast \beta\alpha^\ast
$$

\section{Predictive Parsing}
Backtracking can be avoided if it is possible to detect alternative rule among $A \to \alpha_1 | \alpha_2 | \ldots | \alpha_n$ has to be applied, by considering the current input symbol.

\section{$LL(1) Grammars$}
A grammar $G$ is $LL(1)$ if its predictive parsing table has no multiply-defined entries.
No ambiguous or left-recursive grammar can be $LL(1)$.

An $LL(1)$ parser:
\begin{itemize}
	\item scans the input from left to right;
	\item constructs a leftmost derivation;
	\item uses one lookahead input symbol in making parsing decision.
\end{itemize}

The class of languages that can be parsed using $LL(1)$ parser is a proper subset of the deterministic CFL parser into a program implementing an LL parser.